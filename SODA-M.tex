\documentclass[a4paper, 12pt]{article} % Font size (can be 10pt, 11pt or 12pt) and paper size (remove a4paper for US letter paper)

\usepackage{graphicx} % Required for including pictures
\usepackage{apacite}
\usepackage{sectsty}
\usepackage[margin=1in]{geometry}
\usepackage[T1]{fontenc}
\usepackage{newtxmath,newtxtext}
 \usepackage{enumitem}
   \usepackage{listings}
    \lstset{
   	basicstyle=\ttfamily,
   	mathescape
   }
 \usepackage{xcolor}

  \newcommand\mycomment[1]{\textcolor{red}{\textbf{\textit{(#1)}}}}
 \newcommand\mycommentn[1]{\textcolor{red}{\textbf{\textit{#1}}}\newline}
 

\makeatletter

\renewcommand{\maketitle}{ % Customize the title - do not edit title and author name here, see the TITLE block below
\begin{flushleft} 
{\large\@title\footnotemark[1]} % Increase the font size of the title

\vspace{20pt} % Some vertical space between the title and author name

{\large\@author} % Author name
\end{flushleft}
}


\sectionfont{\fontsize{12}{15}\selectfont\bfseries}
\subsectionfont{\fontsize{12}{15}\selectfont\bfseries\itshape}

%----------------------------------------------------------------------------------------
%	TITLE
%----------------------------------------------------------------------------------------

\title{\textbf{SODA-m} \mycomment{to update}} % Title
\author{Mohan Krishnamoorthy$^{a,2}$, Alexander Brodsky$^a$, and Daniel A. Menasc\'e$^{a}$} % Author


%----------------------------------------------------------------------------------------

\begin{document}

\maketitle % Print the title section

\begin{flushleft} 
\vspace{10pt}
$^a$\textit{Department of Computer Science, George Mason University, Fairfax, Virginia 22030, USA.
% Telephone No.: (703) 993-1537
}\\
\vspace{20pt}
$^a$\{mkrishn4, brodsky, menasce\}@gmu.edu \\
\vspace{20pt}
\textbf{Acknowledgement}\newline
This work was partially supported by NIST Grant No. 70NANB12H277. \newline
\vspace{20pt}
%% To
%{\tiny Provide short biographical notes on all contributors here if the journal requires them\newline Word count: 5691}
\footnotetext[1]{Word Count: 6319 \mycomment{to update}}
\footnotetext[2]{Corresponding Author}
\end{flushleft} 

\newpage
{\large \@title }
\vspace{10pt}
%----------------------------------------------------------------------------------------
%	ABSTRACT AND KEYWORDS
%----------------------------------------------------------------------------------------


\begin{abstract}{\small\noindent
We consider a process with feasibility constraints and metrics of interest including a cost function where the metrics are stochastic functions of the process controls. We propose an efficient stochastic optimization algorithm for the problem of finding process controls that minimize the expectation of cost while satisfying deterministic feasibility constraints and multiple stochastic feasibility constraints with a given high probability. The proposed algorithm is based on (1) a series of deterministic approximations to produce a candidate set of near-optimal control settings for the production process, and (2) stochastic simulations on the candidate set using optimal simulation budget allocation methods. In an experimental study, we demonstrate the proposed algorithm on a use case of a real-world heat-sink service network that involves contract suppliers and manufacturers as well as unit manufacturing processes of shearing, milling, drilling, and machining. The experimental study shows that the proposed algorithm significantly outperforms four popular simulation-based stochastic optimization algorithms.
}
\end{abstract}

{\small  Keywords: decision support;
	decision guidance;
	deterministic approximations;
	stochastic simulation optimization;
	heuristic algorithm } % Keywords

\vspace{5pt} % Some vertical space between the abstract and first section

\section{Introduction}

%Motivation
This paper considers a process with feasibility constraints and metrics of interest including a cost function where the metrics are stochastic functions of the process controls.
This paper is concerned with the development of a one-stage stochastic optimization algorithm for the problem of finding process controls that minimize the expectation of cost while satisfying deterministic feasibility constraints and multiple stochastic feasibility constraints with a given high probability.
These problems are prevalent in manufacturing processes, such as assembly lines and supply chain management where the goal is to find the process controls that minimize the expected cost subject to satisfying deterministic control bound constraints and stochastic steady state demand for the multiple output products with a given high probability. 
There is an increasing need for process analysis and optimization to solve this problem efficiently as companies want to be competitive and need to reduce their cost and improve efficiency of operations in the face of increased global competition. 


%Research Gap
%Current state of the art and its limitations (funnel)
%%high level 
%Simulation based approaches
Stochastic optimization have typically been performed using simulation-based optimization techniques (see \cite{Amaran2016} and \cite{Nguyen2014} for a review of such techniques). 
Tools like SIMULINK \cite{Dabney:2001:MS:557989} and Modelica \cite{elmqvist1998modelica,Provan2012modelica} allow users to run stochastic simulations on models of complex systems in mechanical, hydraulic, thermal, control, and electrical power.
Tools like OMOptim \cite{OMOptim}, Efficient Traceable Model-Based Dynamic Optimization (EDOp) \cite{EDOp}, and jMetal \cite{jMetal} use simulation models to heuristically-guide a trial and error search for the optimal answer. 
However, the general limitation of simulation-based approaches is that simulation is used as a black box, and the underlying mathematical structure is not utilized. 


%NP

From the work on Mathematical Programming (MP), we know that, for deterministic problems, utilizing the mathematical structure can lead to significantly better results in terms of optimality of results and computational complexity compared to simulation-based approaches (see e.g., \cite{Amaran2016} and \cite{Klemmt2009}). 
For this reason, a number of approaches have been developed to bridge the gap between stochastic simulation and MP.
%key things that they do
For instance, \cite{thompson_integrated_1990} propose an integrated approach that combines simulation with MP where the MP problem is constructed from the original stochastic problem with uncertainties being resolved to their mean values by using a sample of black-box simulations. This strategy of extracting an MP from the original problem is also used by \cite{paraskevopoulos_robust_1991} to solve the optimal capacity planning problem by incorporating the original objective function augmented with a penalty on the sensitivity of the objective function to various types of uncertainty.
The authors of \cite{Xu2014MultiFid} propose an ordinal transformation framework, 
consisting of a two-stage optimization framework that first extracts a low fidelity model using simulation or a queuing network model using assumptions that simplify the original problem and then uses this model to reduce the search space over which high fidelity simulations are run to find the optimal solution to the original problem.
%This low fidelity model is in a closed mathematical form and solving this model reduces the search space. High fidelity simulations  are then used on this smaller search space to find the optimal solutions to the original problem.
Other stochastic optimization approaches in the literature try to extract the mathematical structure of the original problem using similar techniques.
However, extraction of the mathematical structure through sampling using a black-box simulation is computationally expensive, especially for real-world processes composed of complex process networks.

%NP
Instead of extracting the mathematical structure using black-box simulation, in \cite{Krishnamoorthy2015}, we used the extraction of mathematical structure from a white-box simulation code analysis as part of a heuristic algorithm to solve a stochastic optimization problem of finding controls for temporal production processes with inventories as to minimize the total cost while satisfying the stochastic demand with a predefined probability.
Similar to the  previous approaches, the mathematical structure was used for approximating a candidate set of solutions by solving a series of deterministic MP problems that approximate the stochastic simulation. 
However, the class of problems considered in \cite{Krishnamoorthy2015} is limited to processes described using piece-wise linear arithmetic. 
Whereas, many processes have models based on physics-based equations with non-linear arithmetic. 

To close this gap, we extended the heuristic algorithm from \cite{Krishnamoorthy2015} to an algorithm called Stochastic Optimization Algorithm based on Deterministic Approximations (SODA) to solve the stochastic optimization problem over a composite service network that involve processes described using non-linear arithmetic \cite{GMU-CS-TR-2017-2}.
However, SODA was only designed for problems with processes that involved stochastic constraint over a single output metric. Whereas, many processes, particularly in manufacturing, have multiple outputs and the stochastic optimization problem needs to consider the satisfaction of the stochastic constraint over all these output metrics. 
Hence, in this paper, we generalize SODA to close the gap for stochastic optimization problems for processes that have feasibility constraints over multiple stochastic metrics and are described using non-linear arithmetic.

%Key contibutions
More specifically, the contributions of this paper are two-fold:
First, we propose a heuristic algorithm called Stochastic Optimization Algorithm based on Deterministic Approximations for multiple metrics (SODA-m) to solve the problem of finding process controls that minimize the expectation of cost while satisfying the deterministic feasibility constraints and multiple stochastic feasibility constraints with a given high probability. 
The proposed algorithm is based on (1) a series of deterministic approximations to produce a candidate set of near-optimal control settings for the production process, and (2) stochastic simulations on the candidate set using optimal simulation budget allocation methods (e.g., see \cite{Chen2011}, \cite{Lee2012OCBACO}).  
%Second, we demonstrate the proposed algorithm on a use case of a real-world heat-sink production process that involves contract suppliers and manufacturers as well as unit manufacturing processes of shearing, milling, drilling, and machining with models from the literature that use non-linear physics-based equations.
Second, we conduct an initial experimental study over a real world use case of heat-sink service network to compare the proposed algorithm with four popular simulation-based stochastic optimization algorithms viz., Nondominated Sorting Genetic Algorithm 2 (NGSA2) \cite{ngsa2}, Indicator Based Evolutionary Algorithm (IBEA) \cite{ibea}, Strength Pareto Evolutionary Algorithm 2 (SPEA2) \cite{spea2}, and Speed-constrained Multi-objective Particle swarm optimization (SMPSO) \cite{NDG09}.
The experimental study demonstrates that SODA-m \mycomment{complete this after running experiments}
% significantly outperforms the other algorithms in terms of optimality of results and computation time. In particular, 
%running over a 12-process problem using a 8-core server with 16GB RAM,
%in 40 minutes, SODA achieves a production  cost lower than that of competing algorithms by 61\%; in 16 hours SODA achieves 29\% better cost; and, in 3 days it achieves 7\% better cost.

%Organization
\mycommentn{write organization of the paper}
%The rest of this paper is organized as follows. Section \ref{sec:prob} formally describes the stochastic optimization problem over steady-state production processes. SODA, including deterministic approximations, is presented in section \ref{sec:algo}. Section \ref{sec:expProbSetup} describes the model of a real world manufacturing use case of a heat-sink service network, which is used in the experimental study presented in section \ref{sec:expResults}. Key observations and extensions are discussed in section \ref{sec:Discussion}. Section \ref{sec:LitRev} further discusses related work. Finally, section \ref{sec:conclusion} concludes with some future research directions. 



\section{Stochastic Optimization over Processes with Multiple Stochastic Feasibility Constraints and Closed-form Non-linear Arithmetic}
\label{sec:prob}

We now borrow the stochastic optimization problem from \cite{GMU-CS-TR-2017-2} and extend it for  processes that have feasibility constraints over multiple stochastic metrics and are described using non-linear arithmetic.

The stochastic optimization problem for such processes assumes a stochastic closed-form arithmetic (SCFA) simulation of the following form.
A SCFA simulation on input variable $\vec{X}$ is a sequence $y_1=expr_1,\dots,y_n=expr_n$
\newline where $expr_i, 1\le i \le n$ is either
\begin{enumerate}[label=(\alph*)]
	\item An arithmetic or boolean expression in terms of a subset of the elements of $\vec{X}$ and/or $y_1,\dots,y_{i-1}$. We say that $y_i$ is arithmetic or boolean if the $expr_i$ is arithmetic or boolean correspondingly.
	\item An expression invoking $PD(\vec{P})$, which is a function that draws from a probability distribution (e.g., gaussian, exponential, uniform) using parameters $\vec{P}$ that are a subset of the elements of $\vec{X}$ and/or $y_1,\dots,y_{i-1}$.
\end{enumerate}
%y_i is deterministic or stochastic
We say that $y_i, 1\le i \le n$ is stochastic if, recursively,
\begin{enumerate}[label=(\alph*)]
	\item $expr_i$ invokes $PD(\vec{P})$, or
	\item $expr_i$ uses at least one stochastic variable $y_j, 1\le j < i$
\end{enumerate}
If $y_i$ is not stochastic, we say that it is deterministic.
%SCFAI computes variable v if v is y_i for i in 1,...,n
Also, we say that a SCFA simulation $\mathbb{S}$  computes a variable \textit{v} if $v=y_i$, for some $1\le i \le n$.

\noindent To clarify, consider a simple SCFA simulation example that consists of the following sequence of expressions: 
\begin{lstlisting}
1: stochSpeed := speed + $\mathcal{N}$(0,$\sigma$)
2: stochTime := $f$(stochSpeed)
3: throughput := 1/stochTime
4: cost := throughput $\times$ pricePerUnit
5: C := lb $\le$ speed $\le$ ub
\end{lstlisting}
In this example, the SCFA simulation computes the stochastic arithmetic variables of \textit{cost} and \textit{throughput} as well as the deterministic boolean variable \textit{C}.
The variable \textit{speed} is deterministic (e.g., machine speed) and it should be bounded within some lower bound (\textit{lb}) and upper bound (\textit{ub}).
The boolean variable \textit{C} describes whether \textit{speed} is bounded.
Also, the effects of \textit{speed} is stochastic (\textit{stochSpeed}) due to normally distributed random noise $\mathcal{N}(0,\sigma)$. 
For the sake of brevity, say that the stochastic time to produce one item (\textit{stochTime}) is obtained from a function described in terms of \textit{stochSpeed}.
Then, \textit{throughput} is computed as the inverse of \textit{stochTime} and \textit{cost} is computed as the product of \textit{throughput} and a fixed parameter of price to produce one unit of item (\textit{pricePerUnit}).

%Our problem (with C, cost, thru)
This paper considers the stochastic optimization problem of finding process controls that minimize the cost expectation while satisfying  deterministic process constraints and steady state demand for the output product with a given probability.
More formally, the stochastic optimization problem is of the form:
\begin{equation}
\label{eq:stochProb}
\begin{aligned}
& \underset{\vec{X}\in\vec{D}}{\text{minimize}}
& & E(\textit{cost}(\vec{X})) \\
& \text{subject to}
& & C(\vec{X}) \wedge \\
&&& P(\textit{thru}(\vec{X}) \ge \theta) \ge \alpha
\end{aligned}
\end{equation}
where $\vec{D} = D_1 \times \dots \times D_n$ is the domain for decision variables $\vec{X}$\\
\indent\indent\indent$\vec{X}$ is a vector of decision variables that range over $\vec{D}$\\
\indent\indent\indent\textit{cost}$(\vec{X})$ is a random variable defined in terms of $\vec{X}$\\
\indent\indent\indent\textit{thru}$(\vec{X})$ is a random variable defined in terms of $\vec{X}$\\
\indent\indent\indent$C(\vec{X})$ is a deterministic constraint in $\vec{X}$ i.e., a function $C:\vec{D} \rightarrow \{true,false\}$\\
\indent\indent\indent$\theta \in \mathbb{R}$ is a throughput threshold\\
\indent\indent\indent$\alpha \in [0,1]$ is a probability threshold, and\\
\indent\indent\indent$P(\textit{thru}(\vec{X}) \ge \theta)$ is the probability that $\textit{thru}(\vec{X})$ is greater than or equal to $\theta$

Note in this problem that upon increasing $\theta$ to some $\theta'$, the size of the space of alternatives that satisfy the stochastic demand constraint in equation \ref{eq:stochProb} increases and hence it can be said that the best solution, i.e., the minimum expected cost is monotonically improving in $\theta'$.
We assume that the random variables, \textit{cost($\vec{X}$)} and \textit{thru($\vec{X}$)} as well as the deterministic constraint $C(\vec{X})$ are expressed by an SCFA simulation $\mathbb{S}$ that computes the stochastic arithmetic variable \textit{(cost, thru)} $ \in \vec{\mathbb{R}}^2$ as well as the deterministic boolean variable $C \in \{true, false\}$. 

While we exemplified the SCFA simulation using a very simple example in this section, many complex real-world processes can be formulated as SCFA simulation such as the use case described in section \ref{sec:expProbSetup}.

%While the SCFA simulation applies to complex real-wold production processes as shown in the use case described in section \ref{sec:expProbSetup}, to make the problem formulation clear, the SCFA simulation of a simple process is presented as an example here.


%
%
%
%\subsection{Convergence of Solutions Involving Monte Carlo Simulation-based Optimization}
%\label{ssec:convergenceProp}
%In order to solve the the stochastic optimization problem described in equation \ref{eq:stochProb}, this paper proposes a simulation based optimization approach with deterministic approximation and heuristics. This approach incorporates the deterministic approximation procedure described in section \ref{ssec:detApproxModel} and then adds noise to the variables of the models to run simulation procedure based on Monte Carlo simulation described in section \ref{ssec:simProc}. In this section, we will show that when the mean of the noise distribution is 0, the noise becomes negligible when the number of simulations is very large due to which it is possible for simulation based optimization approaches to eventually converge to the optimal point. 
%
%If \textit{cost} is the total cost of production as obtained from the deterministic abstraction of the original stochastic problem, then, for \textit{n} sampled values of $\overrightarrow{\mathcal{N}}$, $E(\widehat{cost})_n$ is the expected value of the sample cost. Say that $E(\widehat{cost})$ is its true (population) expected value and suppose that it exist. Since all n sampled values of $\overrightarrow{\mathcal{N}}$ are independent and identically distributed with the same distribution as in $\overrightarrow{\mathcal{N}}$, we can claim that due to the weak law of large numbers, for any $\epsilon > 0$,
%\begin{equation}
%\label{eq:weakProp}
%\begin{aligned}
%& \lim_{n\to\infty} P(|E(\widehat{cost})_n - E(\widehat{cost})| \le \epsilon) = 1
%\end{aligned}
%\end{equation}
%Going further, due to strong law of large numbers, it can be shown that for very large n, the sample mean will eventually converge towards its true mean and stay that way forever (even if we further increase n) i.e., the absolute error will be 0,
%\begin{equation}
%\label{eq:strongProp}
%\begin{aligned}
%& P(\lim_{n\to\infty} |E(\widehat{cost})_n - E(\widehat{cost})| = 0) = 1
%\end{aligned}
%\end{equation}
% Therefore, as n $\rightarrow \infty$, absolute error $\rightarrow 0$. Thus the sample cost will eventually converge towards the true cost as n $\rightarrow \infty$
%\begin{equation}
%\label{eq:ExpCostEq}
%\begin{aligned}
%& E(\widehat{cost})_n = E(\widehat{cost})
%\end{aligned}
%\end{equation}
%By central limit theorem, it is known that the absolute error ($E(\widehat{cost})_n - E(\widehat{cost})$) can be approximately described using a Gaussian noise $\mathcal{N}(0,\frac{SD(\widehat{cost})}{\sqrt{n}})$, where $SD(\widehat{cost})$ is the true (population) standard deviation. Assuming that the error is unbiased and since Monte Carlo simulations typically use large values of n, the true variance estimate will be close to the expected sample variance of the cost. Thus the absolute error can be described as $\mathcal{N}(0,\frac{SD(\widehat{cost})_n}{\sqrt{n}})$. This noise will eventually become negligible with its variance $\rightarrow 0$ as n $\rightarrow \infty$. With these results and assuming that the mean of the noise distribution is 0, we can say that the expected cost contains negligible noise and hence it can be written as:
%
%\begin{equation}
%\label{eq:costEq}
%\begin{aligned}
%& E(\widehat{cost}_n)|_{n \rightarrow \infty} =E(\widehat{cost}) = cost 
%\end{aligned}
%\end{equation}
%Similarly, we can show that:
%\begin{equation}
%\label{eq:thruEq}
%\begin{aligned}
%& E(\widehat{thru}_n)|_{n \rightarrow \infty}  = E(\widehat{thru}) = thru 
%\end{aligned}
%\end{equation}
%
%Simulation based optimization approaches that are based on Monte Carlo simulations, like the one used in this paper eventually converge to the optimal point. This is because the set of feasible solutions although very large, is finite and it is possible to enumerate all the possible solutions and apply a very large number of simulations on each solution to obtain its true value. Although, in our approach, we use deterministic approximations to reduce the search space considerably. As we will show through experimental results in section \ref{sec:expResults}, the convergence to (near) optimal solution happens very quickly even for complex production system in our approach.




%----------------------------------------------------------------------------------------
%	BIBLIOGRAPHY
%----------------------------------------------------------------------------------------

\bibliographystyle{apacite} 
\bibliography{Ref}

%----------------------------------------------------------------------------------------

%----------------------------------------------------------------------------------------
% Table(s) with caption(s) (on individual pages)
%\setcounter{table}{0}
%\begin{table}[htbp]
%\caption{Parameters and Metrics for Phase 1 of Car Manufacturing for Figure~\ref{fig:Tesla}. $\lambda = 3$ cars/hour} 
%\centering  
%\begin{tabular}{|l|c|c|c|c|c|c|c|}
%\hline\hline                        
%Machine           & $S_i$ & $\rho_i$ & $T_i$ & $P_{\rm stat}$ & $P_{\rm dyn}$ & $P_{\rm avg}$ & $E_i$\\
%                        & min    &                &   min  & KJ/sec            & KJ/sec            &  KJ/sec    & GJ     \\ \hline\hline
% Uncoiling 1 \&  2  &  4      & 0.20      &  5.0     & 40                 & 30000            & 114        & 0.03 \\           
%\hline 
%Left cutting          & 10    &   0.50      &    20.0     &  60            & 7200           & 72        &  0.09 \\ \hline
%Underbody cutting & 14 & 0.70 & 47.0  &   80               &  4898         & 56.2       & 0.16 \\ \hline
%Front cutting   &       12   & 0.60    & 30.0    & 80               & 6667                & 75.1     &0.14 \\ \hline
%Right cutting &  10 & 0.50           & 20.0   & 60               & 7200            & 72   & 0.09 \\ \hline
%Die Press 1  &   10  &   0.50      & 20.0     &  18               &  2160             &   21.6   &  0.03 \\ \hline
%Die Press 2  &   8  &   0.40      & 13.3     &  20             &  3750            &   30.3 &  0.02 \\ \hline
%Die Press 3  &   9  &   0.45      & 16.4     &  26               &  5926              &   38.5   &  0.04 \\ \hline
%Die Press 4  &   10  &   0.50      & 20.0     &  18               &  3118              &   23.2   &  0.03 \\ \hline
%Die Press 5  &   8  &   0.40      & 13.3     &  32              &  6125        &   48.6  &  0.04\\ \hline
%Die Press 6  &   7  &   0.35      & 10.8    &  16               &  9796             &   37.1  &  0.02 \\ \hline
%Die Press 7  &   8  &   0.40      & 13.3     &  18                &  3375              &   27.2  &  0.02 \\ \hline
%\end{tabular}
%\label{tab:paramQN} 
%\end{table}
%\begin{table}[htbp]
%\caption{Optimal Service Time Values  for Phase 1 of Car Manufacturing for Figure~\ref{fig:Tesla}. $\lambda = 3$ cars/hour} 
%\centering  
%\begin{tabular}{|l|c|c|c|c|c|c|c|}
%\hline\hline                        
%Machine           & $S_i$ & $\rho_i$ & $T_i$ & $P_{\rm stat}$ & $P_{\rm dyn}$ & $P_{\rm avg}$ & $E_i$\\
%                        & min    &                &   min  & KJ/sec            & KJ/sec            &  KJ/sec    & GJ     \\ \hline\hline
% Uncoiling 1 \&  2  &  3.11      & 0.16     &  3.7     & 40                 & 4.97 $\times 10^4$            & 150.5        & 0.033 \\           
%\hline 
%Left cutting          & 3.11    &   0.16      &    3.7     &  60            & 7.45  $\times 10^4$          & 225.5        &  0.050 \\ \hline
%Underbody cutting & 3.11 & 0.16 & 3.7  &   80               &  9.92 $\times 10^4$         & 300.5       & 0.066 \\ \hline
%Front cutting   &       3.11   & 0.16    & 3.7    & 80               & 9.95 $\times 10^4$                & 300.9     &0.066 \\ \hline
%Right cutting &  3.11 & 0.16           & 3.7   & 60               & 7.44 $\times 10^4$           & 225.3   & 0.050 \\ \hline
%Die Press 1  &   3.11  &   0.16      & 3.7     &  18               &  2.24 $\times 10^4$            &   67.7   &  0.015 \\ \hline
%Die Press 2  &   3.12  &   0.16      & 3.7     &  20             &  2.47 $\times 10^4$           &   75.0 &  0.017 \\ \hline
%Die Press 3  &   3.80  &   0.19      & 4.7     &  26               &  3.32 $\times 10^4$             &   97.4   &  0.027 \\ \hline
%Die Press 4  &   3.70  &   0.19      & 4.5     &  18               &  2.28 $\times 10^4$              &   67.3   &  0.018 \\ \hline
%Die Press 5  &   3.14  &   0.16      & 3.7     &  32              &  3.98 $\times 10^4$       &   120.3  &  0.027\\ \hline
%Die Press 6  &   4.76  &   0.24      & 6.2    &  16               &  2.12 $\times 10^4$             &   59.7  &  0.022 \\ \hline
%Die Press 7  &   3.10  &   0.16      & 3.7     &  18                &  2.25 $\times 10^4$              &   67.9  &  0.015 \\ \hline
%\end{tabular}
%\label{tab:opt} 
%\end{table}
%
%\newpage
%
%
%%----------------------------------------------------------------------------------------
%%----------------------------------------------------------------------------------------
%% Figure caption(s) (as a list) .
%\begin{enumerate}
%\item Figure 1: Autonomic Computing Paradigm
%\item Figure 2: Process Model for Phase 1 of Car Manufacturing
%\item Figure 3: QN for Process Model for Phase 1 of Car Manufacturing
%\item Figure 4: Left y-axis: Completion Time (in minutes) and Right y-axis: Energy Consumed Per Car (in GJoules) vs. Throughput (in cars/hr)
%\end{enumerate}
%
%%ACTUAL FIGURES
%
%\begin{figure}[htbp]
%  \centering
% \includegraphics[width=0.85\textwidth]{Figure1.eps}
%      \caption{Autonomic Computing Paradigm}
%        \label{fig:AC}
%\end{figure}
%
%\begin{figure}[htbp]
%  \centering
% \includegraphics[width=0.85\textwidth]{Figure2.eps}
%      \caption{Process Model for Phase 1 of Car Manufacturing}
%        \label{fig:Tesla}
%\end{figure}
%
%\begin{figure}[htbp]
%  \centering
% \includegraphics[width=1\textwidth]{Figure3.eps}
%      \caption{QN for Process Model for Phase 1 of Car Manufacturing}
%        \label{fig:TeslaQN}
%\end{figure}
%
%\begin{figure}[hbtp]
%\begin{center}
%\includegraphics[scale=0.45]{Figure4.eps}
%\caption{Left y-axis: Completion Time (in minutes) and Right y-axis: Energy Consumed Per Car (in GJoules) vs. Throughput (in cars/hr)}
%\label{fig:comptime-energy}
%\end{center}
%\end{figure}
%%----------------------------------------------------------------------------------------


\end{document}